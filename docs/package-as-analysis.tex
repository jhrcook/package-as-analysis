\documentclass[]{book}
\usepackage{lmodern}
\usepackage{amssymb,amsmath}
\usepackage{ifxetex,ifluatex}
\usepackage{fixltx2e} % provides \textsubscript
\ifnum 0\ifxetex 1\fi\ifluatex 1\fi=0 % if pdftex
  \usepackage[T1]{fontenc}
  \usepackage[utf8]{inputenc}
\else % if luatex or xelatex
  \ifxetex
    \usepackage{mathspec}
  \else
    \usepackage{fontspec}
  \fi
  \defaultfontfeatures{Ligatures=TeX,Scale=MatchLowercase}
\fi
% use upquote if available, for straight quotes in verbatim environments
\IfFileExists{upquote.sty}{\usepackage{upquote}}{}
% use microtype if available
\IfFileExists{microtype.sty}{%
\usepackage{microtype}
\UseMicrotypeSet[protrusion]{basicmath} % disable protrusion for tt fonts
}{}
\usepackage[margin=1in]{geometry}
\usepackage{hyperref}
\hypersetup{unicode=true,
            pdftitle={Package as Analysis},
            pdfauthor={Joshua H. Cook},
            pdfborder={0 0 0},
            breaklinks=true}
\urlstyle{same}  % don't use monospace font for urls
\usepackage{natbib}
\bibliographystyle{apalike}
\usepackage{color}
\usepackage{fancyvrb}
\newcommand{\VerbBar}{|}
\newcommand{\VERB}{\Verb[commandchars=\\\{\}]}
\DefineVerbatimEnvironment{Highlighting}{Verbatim}{commandchars=\\\{\}}
% Add ',fontsize=\small' for more characters per line
\usepackage{framed}
\definecolor{shadecolor}{RGB}{248,248,248}
\newenvironment{Shaded}{\begin{snugshade}}{\end{snugshade}}
\newcommand{\KeywordTok}[1]{\textcolor[rgb]{0.13,0.29,0.53}{\textbf{#1}}}
\newcommand{\DataTypeTok}[1]{\textcolor[rgb]{0.13,0.29,0.53}{#1}}
\newcommand{\DecValTok}[1]{\textcolor[rgb]{0.00,0.00,0.81}{#1}}
\newcommand{\BaseNTok}[1]{\textcolor[rgb]{0.00,0.00,0.81}{#1}}
\newcommand{\FloatTok}[1]{\textcolor[rgb]{0.00,0.00,0.81}{#1}}
\newcommand{\ConstantTok}[1]{\textcolor[rgb]{0.00,0.00,0.00}{#1}}
\newcommand{\CharTok}[1]{\textcolor[rgb]{0.31,0.60,0.02}{#1}}
\newcommand{\SpecialCharTok}[1]{\textcolor[rgb]{0.00,0.00,0.00}{#1}}
\newcommand{\StringTok}[1]{\textcolor[rgb]{0.31,0.60,0.02}{#1}}
\newcommand{\VerbatimStringTok}[1]{\textcolor[rgb]{0.31,0.60,0.02}{#1}}
\newcommand{\SpecialStringTok}[1]{\textcolor[rgb]{0.31,0.60,0.02}{#1}}
\newcommand{\ImportTok}[1]{#1}
\newcommand{\CommentTok}[1]{\textcolor[rgb]{0.56,0.35,0.01}{\textit{#1}}}
\newcommand{\DocumentationTok}[1]{\textcolor[rgb]{0.56,0.35,0.01}{\textbf{\textit{#1}}}}
\newcommand{\AnnotationTok}[1]{\textcolor[rgb]{0.56,0.35,0.01}{\textbf{\textit{#1}}}}
\newcommand{\CommentVarTok}[1]{\textcolor[rgb]{0.56,0.35,0.01}{\textbf{\textit{#1}}}}
\newcommand{\OtherTok}[1]{\textcolor[rgb]{0.56,0.35,0.01}{#1}}
\newcommand{\FunctionTok}[1]{\textcolor[rgb]{0.00,0.00,0.00}{#1}}
\newcommand{\VariableTok}[1]{\textcolor[rgb]{0.00,0.00,0.00}{#1}}
\newcommand{\ControlFlowTok}[1]{\textcolor[rgb]{0.13,0.29,0.53}{\textbf{#1}}}
\newcommand{\OperatorTok}[1]{\textcolor[rgb]{0.81,0.36,0.00}{\textbf{#1}}}
\newcommand{\BuiltInTok}[1]{#1}
\newcommand{\ExtensionTok}[1]{#1}
\newcommand{\PreprocessorTok}[1]{\textcolor[rgb]{0.56,0.35,0.01}{\textit{#1}}}
\newcommand{\AttributeTok}[1]{\textcolor[rgb]{0.77,0.63,0.00}{#1}}
\newcommand{\RegionMarkerTok}[1]{#1}
\newcommand{\InformationTok}[1]{\textcolor[rgb]{0.56,0.35,0.01}{\textbf{\textit{#1}}}}
\newcommand{\WarningTok}[1]{\textcolor[rgb]{0.56,0.35,0.01}{\textbf{\textit{#1}}}}
\newcommand{\AlertTok}[1]{\textcolor[rgb]{0.94,0.16,0.16}{#1}}
\newcommand{\ErrorTok}[1]{\textcolor[rgb]{0.64,0.00,0.00}{\textbf{#1}}}
\newcommand{\NormalTok}[1]{#1}
\usepackage{longtable,booktabs}
\usepackage{graphicx,grffile}
\makeatletter
\def\maxwidth{\ifdim\Gin@nat@width>\linewidth\linewidth\else\Gin@nat@width\fi}
\def\maxheight{\ifdim\Gin@nat@height>\textheight\textheight\else\Gin@nat@height\fi}
\makeatother
% Scale images if necessary, so that they will not overflow the page
% margins by default, and it is still possible to overwrite the defaults
% using explicit options in \includegraphics[width, height, ...]{}
\setkeys{Gin}{width=\maxwidth,height=\maxheight,keepaspectratio}
\IfFileExists{parskip.sty}{%
\usepackage{parskip}
}{% else
\setlength{\parindent}{0pt}
\setlength{\parskip}{6pt plus 2pt minus 1pt}
}
\setlength{\emergencystretch}{3em}  % prevent overfull lines
\providecommand{\tightlist}{%
  \setlength{\itemsep}{0pt}\setlength{\parskip}{0pt}}
\setcounter{secnumdepth}{5}
% Redefines (sub)paragraphs to behave more like sections
\ifx\paragraph\undefined\else
\let\oldparagraph\paragraph
\renewcommand{\paragraph}[1]{\oldparagraph{#1}\mbox{}}
\fi
\ifx\subparagraph\undefined\else
\let\oldsubparagraph\subparagraph
\renewcommand{\subparagraph}[1]{\oldsubparagraph{#1}\mbox{}}
\fi

%%% Use protect on footnotes to avoid problems with footnotes in titles
\let\rmarkdownfootnote\footnote%
\def\footnote{\protect\rmarkdownfootnote}

%%% Change title format to be more compact
\usepackage{titling}

% Create subtitle command for use in maketitle
\providecommand{\subtitle}[1]{
  \posttitle{
    \begin{center}\large#1\end{center}
    }
}

\setlength{\droptitle}{-2em}

  \title{Package as Analysis}
    \pretitle{\vspace{\droptitle}\centering\huge}
  \posttitle{\par}
    \author{Joshua H. Cook}
    \preauthor{\centering\large\emph}
  \postauthor{\par}
      \predate{\centering\large\emph}
  \postdate{\par}
    \date{2019-03-12}

\usepackage{booktabs}

\begin{document}
\maketitle

{
\setcounter{tocdepth}{1}
\tableofcontents
}
\chapter*{Welcome}\label{welcome}
\addcontentsline{toc}{chapter}{Welcome}

This is a manual on how to use the standard R package framework for data
analysis. Though potentially more work, especially at the start, the
purpose of using the R package framework is to maintain a clear and
reproducible analysis.

\textbf{{[}This book is currently in progress, though any
\href{https://github.com/jhrcook/package-as-analysis/issues}{feedback}
is welcome.{]}}

\subsection*{Why}\label{why}
\addcontentsline{toc}{subsection}{Why}

Why bother with maintaining a package framework while also doing data
analysis? It is a great question, especially when one considers the
complexity and fluidity of an analysis and the rigidity of the R package
framework. The answer is that some rigidity is needed - but just the
basics. That is what the R package framework provides. There is a place
for everything, though sometimes getting it to work (ie. build and pass
checks) requires a few extra steps.

During my own analyses, I found things were getting much too
disorganized and decentralized. What would start out as exploratory
would morph into a subdirectory graph more complex than the oringinal
parent analysis. Perhaps for more organized people, this is unnecessary,
but for those of us who want order and aren't sure how to get it, this
framework offers a great place to start.

The final (more abstract) reason for using the R package framework is to
battle the current issue of reproducibility. Reproducibility is the
cornerstone of science - if a finding it true, anyone should be able to
replicate it. However, the scientific community has been dealing with an
astounding amount of irreproducibility, most famously documented by the
\href{http://science.sciencemag.org/content/349/6251/aac4716}{Open
Science Collaboration}. If the analysis is organized as an R package,
though, an analysis can be re-run entirely by anyone else familliar with
R. Thus, whether they are collaborators and competitors, anyone should
be able to follow the analysis a scientist publishes.

\subsection*{Advantages}\label{advantages}
\addcontentsline{toc}{subsection}{Advantages}

There are many advantages to using this framework. Here are just a few,
though I am sure you will find there are many others:

\begin{itemize}
\tightlist
\item
  Because this is a standard framework, other will be able to navigate
  the directories and files adeptly.
\item
  The implementation of tests on functions and subroutines will make
  bugs easier to find and increase overall confidence in the validity of
  the analysis
\item
  This is a seamless mixture of scripts and markdown files for the
  separation of functions and analysis
\item
  Complete documentation of functions makes returning to code later much
  easier!
\item
  The analysis can take advantage of normal R package tools such as
  \href{https://travis-ci.org}{Travis-CI} and
  \href{https://codecov.io}{Codecov} integration,
  \href{https://pkgdown.r-lib.org}{pkgdown}, and
  \href{https://devtools.r-lib.org}{devtools} (build checks,
  documentation, etc.).
\end{itemize}

\subsection*{Examples}\label{examples}
\addcontentsline{toc}{subsection}{Examples}

{[}coming soon{]} Allele-specific \emph{KRAS} copy number alteration

\section*{About}\label{about}
\addcontentsline{toc}{section}{About}

\subsection*{About this Book}\label{about-this-book}
\addcontentsline{toc}{subsection}{About this Book}

{[}TODO{]}

\subsection*{About the Author}\label{about-the-author}
\addcontentsline{toc}{subsection}{About the Author}

I am a classically-trained biologist-turned computational biologist. I
graduated with degrees in Molecular Biology and Biochemistry, and
Chemistry from the University of California, Irvine in 2017. My research
focused on investigating the patterns and mechanisms of dissemination by
which \emph{Toxoplasma gondii}, an obligate, intracellular parasite,
infects a human host
\href{https://www.ncbi.nlm.nih.gov/pubmed/29295815}{Cook \emph{et al.},
2018}. I started my graduate studies at Harvard Medical School in 2018,
and after rotating in a chemical biology lab and a \emph{Vibrio
cholerae} lab, I finally decided to study cancer using computational
biology. Since then, and continuing still today, I have been learning
computer programming and statistics, trying to catch up to my peers.
Consequently, I have fallen in love with R, especially because of the
\href{https://www.tidyverse.org}{Tyidyverse} and tidy data.

\section*{Resources}\label{resources}
\addcontentsline{toc}{section}{Resources}

The best resource for making R packages is
\href{https://r-pkgs.org/index.html}{R Packages} by Hadley Wickham.

There are some useful R packages you will want to have installed:

\begin{itemize}
\tightlist
\item
  \texttt{devtools} - will do most of the development building and
  checking
\item
  \texttt{roxygen2} - makes all of the documentation
\item
  \texttt{usethis} - for preparing all of the pieces and tools you want
  to include
\item
  \texttt{testthat} - for running tests
\item
  \texttt{kintr} - for compiling all of the Rmarkdown files
\end{itemize}

These can be installed using the following code.

\begin{Shaded}
\begin{Highlighting}[]
\KeywordTok{install.packages}\NormalTok{(}\KeywordTok{c}\NormalTok{(}\StringTok{"devtools"}\NormalTok{, }\StringTok{"roxygen2"}\NormalTok{, }\StringTok{"usethis"}\NormalTok{, }\StringTok{"testthat"}\NormalTok{, }\StringTok{"knitr"}\NormalTok{))}
\end{Highlighting}
\end{Shaded}

\section*{License}\label{license}
\addcontentsline{toc}{section}{License}

This work is under a
\href{http://creativecommons.org/licenses/by-nc-sa/4.0/}{Attribution-NonCommercial-ShareAlike
4.0 International License (CC BY-NC-SA 4.0)}

\chapter{Framework}\label{framework}

\section{Introduction}\label{introduction}

I have decided to introduce the R packagge framework before going
through the set-up process because, quite frankly, I think this is more
important. There are many sites that outline the steps of creating an R
package (they likely do it better than I will next chapter), but the
main point of this book is how to make the R package into a data
analysis project. Therfore, understanding the use of the framework
should be the reader's focus, here.

I try to adhere to the following process for each peace of the
framework:

\begin{enumerate}
\def\labelenumi{\arabic{enumi}.}
\tightlist
\item
  what is it and what is its main role?
\item
  when does it need to be used or adjusted?
\item
  what is unique about its use for a data analysis project?
\end{enumerate}

\section{Vignettes}\label{vignettes}

I am starting with vignettes because they are the heart of the data
analysis. Ironically, they are often covered last in tutorials on
creating R packages, and rightly so! For an R package, the vignettes
merely show the functionality of the package. They are usually user
guides or are attempts to inspire potential users.

However, in a data analysis, the vignettes \emph{are} the analysis. They
are where ideas are formed, discussed, tested, and validated. Everything
is centered around a vignette. If you already use R markdown for data
analysis, then this should be of little perturbance to your standard
operating procedure. If you do not, though, this should be a welcomed
change.

\subsection{Starting a Vignette}\label{starting-a-vignette}

Ideally, each vignette will be a different step in the analysis.
Alternatively, each vignette could be a different part of the project
(e.g.~analyzing the results of a screen in one vignette and analyzing
the results of validation experiments in another). However you choose to
divide up the analysis is your choice (though I have found R markdown
files to slow down RStudio if they get too long or image intensive). To
begin a vignette, use the usethis function, passing the name of the
vignette.

\begin{Shaded}
\begin{Highlighting}[]
\NormalTok{usethis}\OperatorTok{::}\KeywordTok{use_vignette}\NormalTok{(}\StringTok{"0101_part1_firstvignette"}\NormalTok{)}
\CommentTok{#> ✔ Setting active project to '/path/to/pkg/ExampleAnalysisPackage'}
\CommentTok{#> ✔ Adding 'knitr' to Suggests field in DESCRIPTION}
\CommentTok{#> ✔ Setting VignetteBuilder field in DESCRIPTION to 'knitr'}
\CommentTok{#> ✔ Adding 'rmarkdown' to Suggests field in DESCRIPTION}
\CommentTok{#> ✔ Creating 'vignettes/'}
\CommentTok{#> ✔ Adding '*.html', '*.R' to 'vignettes/.gitignore'}
\CommentTok{#> ✔ Adding 'inst/doc' to '.gitignore'}
\CommentTok{#> ✔ Creating 'vignettes/0101_part1_firstvignette.Rmd'}
\CommentTok{#> ● Modify 'vignettes/0101_part1_firstvignette.Rmd'}
\end{Highlighting}
\end{Shaded}

Notice that the ``.Rmd'' extension is ommited in the above function -
usethis adds the appropriate extensions in most cases.

It is important to deliberately decide on a naming scheme. I have opted
for ``(\#\#)(\#\#)\_part\_subpart" where the leading number is made of
two levels followed by the names of the two levels. For the leading
number, the first two digits are the highest level of orgnization and
the second two digits are for the sublevel. For the naming, the first
name (``part'') is the name referring to the layer of organization
dictated by the first two digits of the leading number. The second name
(``subpart'') is referring to the layer of organization dictated by the
second two digits of the leading number.

Therefore, the follow-up analysis that follows
``0101\_part1\_firstvignette'' could be ``0102\_part1\_secondvignette''.
A vignette for another part of the project would alternatively be called
``0201\_part2\_firstvignette''. I now have a two branches of analysis in
my project, the first with two pages of analysis and the second with
only one.

\subsection{Formatting a Vignette}\label{formatting-a-vignette}

{[}TODO{]}

\section{The R/ Directory}\label{the-r-directory}

\subsection{Normal Use}\label{normal-use}

The R directory will hold all R scripts (usually files ending with the
``.R'' extension). For a normal package, this is all of the source code
that builds the package.

\subsection{For Data Analysis}\label{for-data-analysis}

For the purpose of a data analysis project, this will still hold R
scripts (with the ``.R'' extension), but the organization of scripts is
different.

\chapter{Getting Started}\label{getting-started}

Setting up the package is mostly automated and is well documented in
\href{https://r-pkgs.org/index.html}{R Packages} by Hadley Wickham. If
you are running the analysis on your local machine, I would recommend
using \href{https://www.rstudio.com}{RStudio} (which you likely already
do), but this is possible to do on a remote computing cluster (which is
how I work). I begin by going through the steps of setting up the basic
package framework, which is the same for local or remote work. Following
that, there is a section for how I work remotely. This can be skipped if
you only work locally or if you already have a system you enjoy (though
I highly recommend the system I currently use). I finish off with a few
extras that I recommend using, but are not necessary.

\section{Setting Up a R Package}\label{setting-up-a-r-package}

The set up process is rather simple. If using RStudio, you can start a
new R project as a package. Otherwise, the following command will get
the basic framework started. There is a lot of overlap between the
devtools and usethis package. I believe that RStudio is trying to fade
out devtools and instead have people use the various packages that were
split from it, including usethis.

The advantage to using the usethis functions for seemingly simple tasks
(such as making the ``data-raw'' directory) is that it will also add the
necessary lines to ``.RBuildignore,'' the ``DESCRIPTION,'' and
``NAMESPACE'' if needed.

\subsection{The Basics}\label{the-basics}

It's easy to create the package from the R console, just replace the
directory path in the following example with your desired directory.
Usethis will \emph{creat} the ``ExampleAnalysisPackage'' directory - do
not make this directory beforehand.

\begin{Shaded}
\begin{Highlighting}[]
\NormalTok{usethis}\OperatorTok{::}\KeywordTok{create_package}\NormalTok{(}\StringTok{"/path/to/pkg/ExampleAnalysisPackage"}\NormalTok{)}
\CommentTok{#> ✔ Setting active project to 'p/path/to/pkg/ExampleAnalysisPackage'}
\CommentTok{#> ✔ Creating 'R/'}
\CommentTok{#> ✔ Creating 'man/'}
\CommentTok{#> ✔ Writing 'DESCRIPTION'}
\CommentTok{#> ✔ Writing 'NAMESPACE'}
\CommentTok{#> ✔ Changing working directory to '/path/to/pkg/ExampleAnalysisPackage'}
\end{Highlighting}
\end{Shaded}

You can then add a license as shown below. I generally use a GPL-3,
though you can get a lot of information on the common licenses at
\href{https://choosealicense.com}{choosealicense.com}.

\begin{Shaded}
\begin{Highlighting}[]
\NormalTok{usethis}\OperatorTok{::}\KeywordTok{use_gpl3_license}\NormalTok{(}\DataTypeTok{name =} \StringTok{"Your Name"}\NormalTok{)}
\CommentTok{#> ✔ Setting active project to '/path/to/pkg/ExampleAnalysisPackage'}
\CommentTok{#> ✔ Setting License field in DESCRIPTION to 'GPL-3'}
\CommentTok{#> ✔ Writing 'LICENSE.md'}
\CommentTok{#> ✔ Adding '^LICENSE\textbackslash{}\textbackslash{}.md$' to '.Rbuildignore'}
\end{Highlighting}
\end{Shaded}

Prepare the project to use roxygen for documentation.

\begin{Shaded}
\begin{Highlighting}[]
\NormalTok{usethis}\OperatorTok{::}\KeywordTok{use_roxygen_md}\NormalTok{()}
\CommentTok{#> ✔ Setting Roxygen field in DESCRIPTION to 'list(markdown = TRUE)'}
\CommentTok{#> ✔ Setting RoxygenNote field in DESCRIPTION to '6.1.1'}
\CommentTok{#> ● Run `devtools::document()`}
\end{Highlighting}
\end{Shaded}

Create a README file. You can also opt to use a normal Markdown file
with \texttt{usethis::use\_readme\_md()}, though I would recommend to
just go with an R Markdown file.

\begin{Shaded}
\begin{Highlighting}[]
\NormalTok{usethis}\OperatorTok{::}\KeywordTok{use_readme_rmd}\NormalTok{()}
\CommentTok{#> ✔ Writing 'README.Rmd'}
\CommentTok{#> ✔ Adding '^README\textbackslash{}\textbackslash{}.Rmd$' to '.Rbuildignore'}
\CommentTok{#> ● Modify 'README.Rmd'}
\end{Highlighting}
\end{Shaded}

Create a ``NEWS'' file for announcing major changes to the project.

\begin{Shaded}
\begin{Highlighting}[]
\NormalTok{usethis}\OperatorTok{::}\KeywordTok{use_news_md}\NormalTok{()}
\CommentTok{#> ✔ Writing 'NEWS.md'}
\CommentTok{#> ● Modify 'NEWS.md'}
\end{Highlighting}
\end{Shaded}

Create a ``data-raw'' directory.

\begin{Shaded}
\begin{Highlighting}[]
\NormalTok{usethis}\OperatorTok{::}\KeywordTok{use_data_raw}\NormalTok{()}
\CommentTok{#> ✔ Creating 'data-raw/'}
\CommentTok{#> ✔ Adding '^data-raw$' to '.Rbuildignore'}
\CommentTok{#> Next:}
\CommentTok{#> ● Add data creation scripts in 'data-raw/'}
\CommentTok{#> ● Use `usethis::use_data()` to add data to package}
\end{Highlighting}
\end{Shaded}

Finally, set up the use of testthat package for testing.

\begin{Shaded}
\begin{Highlighting}[]
\NormalTok{usethis}\OperatorTok{::}\KeywordTok{use_testthat}\NormalTok{()}
\CommentTok{#> ✔ Adding 'testthat' to Suggests field in DESCRIPTION}
\CommentTok{#> ✔ Creating 'tests/testthat/'}
\CommentTok{#> ✔ Writing 'tests/testthat.R'}
\end{Highlighting}
\end{Shaded}

\textbf{Note:} If you are working remotely (i.e.. SSHing to the computer
running the code), many of the usethis functions will open the file that
you just asked them to create (e.g.. open ``NEWS.md'' after using
\texttt{use\_news\_md()}) in vim. To suppress this, just pass the
parameter \texttt{open\ =\ FALSE}. Otherwise, it is set to
\texttt{interactive()}.

\subsection{Git and GitHub}\label{git-and-github}

If you are programming, you should be using git. This is especially
important in the sciences because git logs can be used to resolve legal
conflicts and issues of data falsification. GitHub is not essential,
though I would highly recommend you use it because it makes managing
files and collaboration much easier. It is also essential for taking
advantage of some of the best parts of an R package such as build checks
and \texttt{pkgdown} (see Extras below)

To get started with git, there are \emph{tons} of resources available,
so I will not describe it here. If you are new to git and GitHub, here
are a few good resources to get you started:

\begin{itemize}
\tightlist
\item
  \href{https://medium.freecodecamp.org/what-is-git-and-how-to-use-it-c341b049ae61}{An
  introduction to Git: what it is, and how to use it}
\item
  \href{https://www.digitalocean.com/community/tutorials/how-to-use-git-a-reference-guide}{How
  To Use Git: A Reference Guide}
\item
  \href{https://guides.github.com/activities/hello-world/}{GitHub
  Guides: Hello World}
\end{itemize}

There is a usethis function to initiate git
(\texttt{usethis::use\_git()}) though I always prefer to set up myself.

\begin{Shaded}
\begin{Highlighting}[]
\NormalTok{usethis}\OperatorTok{::}\KeywordTok{use_git}\NormalTok{()}
\CommentTok{#> ✔ Initialising Git repo}
\CommentTok{#> ✔ Adding '.Rhistory', '.RData', '.Rproj.user' to '.gitignore'}
\CommentTok{#> OK to make an initial commit of 8 files?}
\CommentTok{#> 1: Yeah}
\CommentTok{#> 2: Absolutely not}
\CommentTok{#> 3: No way}
\DecValTok{1}
\CommentTok{#> Selection: 1}
\CommentTok{#> ✔ Adding files and committing}
\end{Highlighting}
\end{Shaded}

\section{Working Remotely}\label{working-remotely}

If you conduct work remotely, I'm going to assume that you have SSH set
up and running. Otherwise, there are plenty of resources available, and
you should review the material available by your system admin.

Though I prefer RStudio for normal package development, I spare my
computer the pain of performing complex and heavy computation, opting
instead to off-load it to the \href{https://rc.hms.harvard.edu}{Harvard
Medical School Research Computing Cluster}. Therefore, I use
\href{https://www.sublimetext.com}{SublimeText3} as by text editor and
send code to the remote computing node over SSH using
\href{https://iterm2colorschemes.com}{iTerm2} as my terminal. Finally, I
use SSH File System (SSHFS) to ``mount'' the remote directory to my
local directory.

\subsection{SublimeText3 Set-Up}\label{sublimetext3-set-up}

Here are the handful of SublimeText3 (ST3) packages I use for R coding,
followed by any particular notes on their use:

\begin{itemize}
\tightlist
\item
  \href{https://packagecontrol.io/packages/LSP}{LSP} - ``Gives Sublime
  Text 3 rich editing features for languages with Language Server
  Protocol support''
\item
  \href{https://packagecontrol.io/packages/MarkdownEditing}{MarkdownEditing}
  - ``Markdown plugin for Sublime Text. Provides\ldots{} more robust
  syntax highlighting and useful Markdown editing features for Sublime
  Text.''
\item
  \href{https://packagecontrol.io/packages/R-IDE}{R-IDE} -
  ``{[}A{]}iming to utilize the use of language server + better support
  R Markdown + better support of R packaging + \ldots{}''
\item
  \href{https://packagecontrol.io/packages/SendCode}{SendCode} - ``Send
  code and text to macOS and Linux Terminals, iTerm, ConEmu, Cmder,
  Tmux, Terminus; R (RStudio), Julia, IPython.''
\end{itemize}

LSP and R-IDE handle syntax and completion in ST3. It isn't a great
system, so if you know of a better set-up in ST3,
\href{https://github.com/jhrcook/package-as-analysis/issues}{please let
me know}. MarkdownEditing and R-IDE combine to make R Markdown feasible.
The SendCode package essentially copies, pastes, and runs my code
written in ST3 to the terminal when I press \texttt{command} +
\texttt{return}. This way, I can type in ST3 and run in the terminal
without using the mouse.

Before moving on, I made this snippet to quickly add a code chunk.

\begin{Shaded}
\begin{Highlighting}[]
\KeywordTok{<snippet>}
        \KeywordTok{<content>}\BaseNTok{<![CDATA[}
\NormalTok{```\{r $\{1:chunk_name\}\}}
\NormalTok{$0}
\NormalTok{```}
\BaseNTok{]]>}\KeywordTok{</content>}
        \CommentTok{<!-- Optional: Set a tabTrigger to define how to trigger the snippet -->}
        \KeywordTok{<tabTrigger>}\NormalTok{rchunk}\KeywordTok{</tabTrigger>}
        \CommentTok{<!-- Optional: Set a scope to limit where the snippet will trigger -->}
        \KeywordTok{<scope>}\NormalTok{text.html.markdown.multimarkdown, text.html.markdown}\KeywordTok{</scope>}
        \KeywordTok{<description>}\NormalTok{create a Rmd code chunk}\KeywordTok{</description>}
\KeywordTok{</snippet>}
\end{Highlighting}
\end{Shaded}

\subsection{Using SSH File System}\label{using-ssh-file-system}

I use SSH File System (SSFHS) to ``mount'' my remote directory to my
local directory. It is essentially SFTP and SSH combined (the details go
right over my head) and it is fairly easy to get set-up. Here is a
\href{https://github.com/osxfuse/osxfuse/wiki/SSHFS}{link} to get
everything going, and I have included the steps I used below.

To start I downloaded and installed
\href{https://osxfuse.github.io}{FUSE for macOS}. Then I downloaded and
installed the \href{https://github.com/osxfuse/sshfs/releases}{latest
build of SSHFS}. Finally, I created an empty directory that will become
the place that I mount the remote directory. Typically, I make the root
of the package, though I could see instances where you would want the
package in a subdirectory below it.

\begin{Shaded}
\begin{Highlighting}[]
\CommentTok{# on local}
\FunctionTok{mkdir}\NormalTok{ ~/local/path/pkgName}
\end{Highlighting}
\end{Shaded}

On the remote server, I make a directory with the same name

\begin{Shaded}
\begin{Highlighting}[]
\CommentTok{# on remote}
\FunctionTok{mkdir}\NormalTok{ /remote/path/pkgName}
\end{Highlighting}
\end{Shaded}

Finally, to connect the two, I use the following command that is pretty
much identical to initiating a normal SSH session.

\begin{Shaded}
\begin{Highlighting}[]
\CommentTok{# on local}
\ExtensionTok{sshfs}\NormalTok{ username@remote.host.com:/remote/path/pkgName ~/local/path/pkgName}
\end{Highlighting}
\end{Shaded}

Now, the computer will treat the mount just like a normal flash drive,
and ST3 fully accepts it. The only change I made to ST3 was to map a
key-binding to ``Project/Refresh Folders''. This way, if new files are
created remotely, a quick key-stroke and everything is visible in the
ST3 sidebar.

\subsection{Git and GitHub}\label{git-and-github-1}

If working remotely, I have found it much easier to handle everything
git-related on the remote side. Therefore, I created SSH RSA keys and
shared the public one with the GitHub repository so I could push over
SSH. Setting this up is pretty simple and well outlined in
\href{https://help.github.com/en/articles/connecting-to-github-with-SSH}{Connecting
to GitHub with SSH}.

\section{Extras}\label{extras}

Though these next few items are not required, I \emph{highly recommend}
implementing them because they each take advantage of the fact that this
project adheres to the standard R framework. Their different functions
are all reasons to go through the trouble of maintaining this framework.

\subsection{pkgdown}\label{pkgdown}

Pkgdown ties a bow around your package, slaps it on the bottom, and
builds a gorgeous and professional website rich with useful features. It
builds the documentation for easy reference, presents the vignettes, and
organizes all of the package meta-data so it is easily viewable and
understandable. Here are some packages that take advantage of pkgdown:

\begin{itemize}
\tightlist
\item
  \href{https://pkgdown.r-lib.org}{pkgdown} (of course)
\item
  \href{https://ggplot2.tidyverse.org}{ggplot2}
\item
  \href{https://nanx.me/ggsci/}{ggsci}
\item
  \href{https://jhrcook.github.io/ggasym/index.html}{ggasym} (a
  shameless plug of my own lil' package)
\end{itemize}

The use of pkgdown obviously begins with a usethis function.

\begin{Shaded}
\begin{Highlighting}[]
\NormalTok{usethis}\OperatorTok{::}\KeywordTok{use_pkgdown}\NormalTok{()}
\CommentTok{#> ● Modify '_pkgdown.yml'}
\CommentTok{#> ✔ Adding '^_pkgdown\textbackslash{}\textbackslash{}.yml$' to '.Rbuildignore'}
\CommentTok{#> ✔ Creating 'docs/'}
\CommentTok{#> ✔ Adding '^docs$' to '.Rbuildignore'}
\end{Highlighting}
\end{Shaded}

All that you have to do from there is use
\texttt{pkgdown::build\_site()} to build the site whenever the project
is at a good stopping point for the day. (If working remotely, pass
\texttt{preview\ =\ FALSE} to prevent pkgdown from searching for a
browser to display in when completed.)

To show the website on GitHub, go to ``Settings'' in the repository, and
select ``master branch /docs folder'' from the options in the ``GitHub
Pages'' section. It should look something like this (another shameless
plug for lil' ole' ggasym).

\subsection{Travis-CI, Appveyor, and
Codecov}\label{travis-ci-appveyor-and-codecov}

\textbf{Note:} To use these, you must have a GitHub repository that you
have push to.

GitHub integration also opens up the use of continuous integration (CI)
apps. \href{https://travis-ci.org}{Travis-CI} and
\href{https://www.appveyor.com}{Appveyor} are useful for checking the
build status of the package. I just use both because they each require
so little effort to integrate and each provides their own suite of
functions. Notably, Appveyor build the package on Linux and Windows. To
get started, just use usethis.

\href{https://codecov.io}{Codecov} provides an indication as to how well
the package's tests cover the code. Though not a perfect measure of test
quality (nothing ever will be), I find this tool to be helpful for me to
find which functions I have and have not created tests for.

\begin{Shaded}
\begin{Highlighting}[]
\NormalTok{usethis}\OperatorTok{::}\KeywordTok{use_travis}\NormalTok{()}
\CommentTok{#> ✔ Setting active project to '/path/to/pkg/ExampleAnalysisPackage'}
\CommentTok{#> ✔ Writing '.travis.yml'}
\CommentTok{#> ✔ Adding '^\textbackslash{}\textbackslash{}.travis\textbackslash{}\textbackslash{}.yml$' to '.Rbuildignore'}
\CommentTok{#> ● Turn on travis for your repo at https://travis-ci.org/profile/jhrcook}
\CommentTok{#> ● Add a Travis build status badge by adding the following line to your README:}
\CommentTok{#> Copying code to clipboard:}
\CommentTok{#>   [![Travis build status](https://travis-ci.org/jhrcook/ExampleAnalysisPackage.svg?branch=master)](https://travis-ci.org/jhrcook/ExampleAnalysisPackage)}
\end{Highlighting}
\end{Shaded}

\begin{Shaded}
\begin{Highlighting}[]
\NormalTok{usethis}\OperatorTok{::}\KeywordTok{use_appveyor}\NormalTok{()}
\CommentTok{#> ✔ Writing 'appveyor.yml'}
\CommentTok{#> ✔ Adding '^appveyor\textbackslash{}\textbackslash{}.yml$' to '.Rbuildignore'}
\CommentTok{#> ● Turn on AppVeyor for this repo at https://ci.appveyor.com/projects/new}
\CommentTok{#> ● Add a AppVeyor build status badge by adding the following line to your README:}
\CommentTok{#> Copying code to clipboard:}
\CommentTok{#>   [![AppVeyor build status](https://ci.appveyor.com/api/projects/status/github/jhrcook/ExampleAnalysisPackage?branch=master&svg=true)](https://ci.appveyor.com/project/jhrcook/ExampleAnalysisPackage)}
\end{Highlighting}
\end{Shaded}

\begin{Shaded}
\begin{Highlighting}[]
\NormalTok{usethis}\OperatorTok{::}\KeywordTok{use_coverage}\NormalTok{(}\StringTok{"codecov"}\NormalTok{)}
\CommentTok{#> ✔ Adding 'covr' to Suggests field in DESCRIPTION}
\CommentTok{#> ✔ Writing 'codecov.yml'}
\CommentTok{#> ✔ Adding '^codecov\textbackslash{}\textbackslash{}.yml$' to '.Rbuildignore'}
\CommentTok{#> ● Add a Coverage status badge by adding the following line to your README:}
\CommentTok{#> Copying code to clipboard:}
\CommentTok{#>   [![Coverage status](https://codecov.io/gh/jhrcook/ExampleAnalysisPackage/branch/master/graph/badge.svg)](https://codecov.io/github/jhrcook/ExampleAnalysisPackage?branch=master)}
\CommentTok{#> ● Add to '.travis.yml':}
\CommentTok{#> Copying code to clipboard:}
\CommentTok{#>   after_success:}
\CommentTok{#>     - Rscript -e 'covr::codecov()'}
\end{Highlighting}
\end{Shaded}

Just follow the instructions printed out to get everything set up. If
this your first time using any of the tools, then you will have to grant
them access to your GitHub repositories, and they will do the rest.

The usethis command will also produce the markdown code for showing the
status badges for each tools. Placing these below the package name in
the README.Rmd is standard practice and will tell pkgdown to put them in
the side bar of the site.

On top of looking good and being informative for you during the
development process, these badges will also provide visitors an
indication as to the quality and maintenance of the package. A few good
badges will likely make visitors more trusting of your results.

\subsection{Spelling}\label{spelling}

I need not explain why a spell check is useful. Shockingly, this is easy
to implement with a usethis function.

\begin{Shaded}
\begin{Highlighting}[]
\NormalTok{usethis}\OperatorTok{::}\KeywordTok{use_spell_check}\NormalTok{()}
\CommentTok{#> ✔ Adding 'spelling' to Suggests field in DESCRIPTION}
\CommentTok{#> ✔ Setting Language field in DESCRIPTION to 'en-US'}
\CommentTok{#> No changes required to /path/to/pkg/ExampleAnalysisPackage/inst/WORDLIST}
\CommentTok{#> Updated /path/to/pkg/ExampleAnalysisPackage/tests/spelling.R}
\CommentTok{#> ● Run `devtools::check()` to trigger spell check}
\end{Highlighting}
\end{Shaded}

From then on, spelling can easily be checking using the
\href{https://github.com/ropensci/spelling}{spelling} package.

\begin{Shaded}
\begin{Highlighting}[]
\CommentTok{# check spelling package-wide}
\NormalTok{spelling}\OperatorTok{::}\KeywordTok{spell_check_package}\NormalTok{()}
\CommentTok{# check spelling in specific files}
\NormalTok{spelling}\OperatorTok{::}\KeywordTok{spell_check_files}\NormalTok{(}\StringTok{"README.Rmd"}\NormalTok{)  }\CommentTok{# ignores code chunks}
\end{Highlighting}
\end{Shaded}

After making corrections, there are likely still words that are correct
but not in the dictionary used by the package. These can be added to a
package-specific word-list.

\begin{Shaded}
\begin{Highlighting}[]
\NormalTok{spelling}\OperatorTok{::}\KeywordTok{update_wordlist}\NormalTok{()}
\end{Highlighting}
\end{Shaded}

To skip the confirmation after the above command, use the parameter
\texttt{confirm\ =\ TRUE}.

There is also the option to integrate spelling into the testing process
such that it runs automatically anytime the package is checked.

\begin{Shaded}
\begin{Highlighting}[]
\NormalTok{spelling}\OperatorTok{::}\KeywordTok{spell_check_setup}\NormalTok{()}
\end{Highlighting}
\end{Shaded}

\section{Conclusion}\label{conclusion}

Setting up a R package is actually pretty easy, flexible, and
customizable. It is also amendable to change later on (especially
because of the usethis package), so there is not need to fret about
making it perfect - pieces can always be added later.

\textbf{Note:} At this point, the package is unlikely to pass checks
because there are no tests.

\chapter{Workflow}\label{workflow}

\chapter{\texorpdfstring{Example: Allele-specific \emph{KRAS}
CNA}{Example: Allele-specific KRAS CNA}}\label{example-allele-specific-kras-cna}

{[}in progress{]}

\bibliography{book.bib}


\end{document}
