\documentclass[]{book}
\usepackage{lmodern}
\usepackage{amssymb,amsmath}
\usepackage{ifxetex,ifluatex}
\usepackage{fixltx2e} % provides \textsubscript
\ifnum 0\ifxetex 1\fi\ifluatex 1\fi=0 % if pdftex
  \usepackage[T1]{fontenc}
  \usepackage[utf8]{inputenc}
\else % if luatex or xelatex
  \ifxetex
    \usepackage{mathspec}
  \else
    \usepackage{fontspec}
  \fi
  \defaultfontfeatures{Ligatures=TeX,Scale=MatchLowercase}
\fi
% use upquote if available, for straight quotes in verbatim environments
\IfFileExists{upquote.sty}{\usepackage{upquote}}{}
% use microtype if available
\IfFileExists{microtype.sty}{%
\usepackage{microtype}
\UseMicrotypeSet[protrusion]{basicmath} % disable protrusion for tt fonts
}{}
\usepackage[margin=1in]{geometry}
\usepackage{hyperref}
\hypersetup{unicode=true,
            pdftitle={Package as Analysis},
            pdfauthor={Joshua H. Cook},
            pdfborder={0 0 0},
            breaklinks=true}
\urlstyle{same}  % don't use monospace font for urls
\usepackage{natbib}
\bibliographystyle{apalike}
\usepackage{color}
\usepackage{fancyvrb}
\newcommand{\VerbBar}{|}
\newcommand{\VERB}{\Verb[commandchars=\\\{\}]}
\DefineVerbatimEnvironment{Highlighting}{Verbatim}{commandchars=\\\{\}}
% Add ',fontsize=\small' for more characters per line
\usepackage{framed}
\definecolor{shadecolor}{RGB}{248,248,248}
\newenvironment{Shaded}{\begin{snugshade}}{\end{snugshade}}
\newcommand{\KeywordTok}[1]{\textcolor[rgb]{0.13,0.29,0.53}{\textbf{#1}}}
\newcommand{\DataTypeTok}[1]{\textcolor[rgb]{0.13,0.29,0.53}{#1}}
\newcommand{\DecValTok}[1]{\textcolor[rgb]{0.00,0.00,0.81}{#1}}
\newcommand{\BaseNTok}[1]{\textcolor[rgb]{0.00,0.00,0.81}{#1}}
\newcommand{\FloatTok}[1]{\textcolor[rgb]{0.00,0.00,0.81}{#1}}
\newcommand{\ConstantTok}[1]{\textcolor[rgb]{0.00,0.00,0.00}{#1}}
\newcommand{\CharTok}[1]{\textcolor[rgb]{0.31,0.60,0.02}{#1}}
\newcommand{\SpecialCharTok}[1]{\textcolor[rgb]{0.00,0.00,0.00}{#1}}
\newcommand{\StringTok}[1]{\textcolor[rgb]{0.31,0.60,0.02}{#1}}
\newcommand{\VerbatimStringTok}[1]{\textcolor[rgb]{0.31,0.60,0.02}{#1}}
\newcommand{\SpecialStringTok}[1]{\textcolor[rgb]{0.31,0.60,0.02}{#1}}
\newcommand{\ImportTok}[1]{#1}
\newcommand{\CommentTok}[1]{\textcolor[rgb]{0.56,0.35,0.01}{\textit{#1}}}
\newcommand{\DocumentationTok}[1]{\textcolor[rgb]{0.56,0.35,0.01}{\textbf{\textit{#1}}}}
\newcommand{\AnnotationTok}[1]{\textcolor[rgb]{0.56,0.35,0.01}{\textbf{\textit{#1}}}}
\newcommand{\CommentVarTok}[1]{\textcolor[rgb]{0.56,0.35,0.01}{\textbf{\textit{#1}}}}
\newcommand{\OtherTok}[1]{\textcolor[rgb]{0.56,0.35,0.01}{#1}}
\newcommand{\FunctionTok}[1]{\textcolor[rgb]{0.00,0.00,0.00}{#1}}
\newcommand{\VariableTok}[1]{\textcolor[rgb]{0.00,0.00,0.00}{#1}}
\newcommand{\ControlFlowTok}[1]{\textcolor[rgb]{0.13,0.29,0.53}{\textbf{#1}}}
\newcommand{\OperatorTok}[1]{\textcolor[rgb]{0.81,0.36,0.00}{\textbf{#1}}}
\newcommand{\BuiltInTok}[1]{#1}
\newcommand{\ExtensionTok}[1]{#1}
\newcommand{\PreprocessorTok}[1]{\textcolor[rgb]{0.56,0.35,0.01}{\textit{#1}}}
\newcommand{\AttributeTok}[1]{\textcolor[rgb]{0.77,0.63,0.00}{#1}}
\newcommand{\RegionMarkerTok}[1]{#1}
\newcommand{\InformationTok}[1]{\textcolor[rgb]{0.56,0.35,0.01}{\textbf{\textit{#1}}}}
\newcommand{\WarningTok}[1]{\textcolor[rgb]{0.56,0.35,0.01}{\textbf{\textit{#1}}}}
\newcommand{\AlertTok}[1]{\textcolor[rgb]{0.94,0.16,0.16}{#1}}
\newcommand{\ErrorTok}[1]{\textcolor[rgb]{0.64,0.00,0.00}{\textbf{#1}}}
\newcommand{\NormalTok}[1]{#1}
\usepackage{longtable,booktabs}
\usepackage{graphicx,grffile}
\makeatletter
\def\maxwidth{\ifdim\Gin@nat@width>\linewidth\linewidth\else\Gin@nat@width\fi}
\def\maxheight{\ifdim\Gin@nat@height>\textheight\textheight\else\Gin@nat@height\fi}
\makeatother
% Scale images if necessary, so that they will not overflow the page
% margins by default, and it is still possible to overwrite the defaults
% using explicit options in \includegraphics[width, height, ...]{}
\setkeys{Gin}{width=\maxwidth,height=\maxheight,keepaspectratio}
\IfFileExists{parskip.sty}{%
\usepackage{parskip}
}{% else
\setlength{\parindent}{0pt}
\setlength{\parskip}{6pt plus 2pt minus 1pt}
}
\setlength{\emergencystretch}{3em}  % prevent overfull lines
\providecommand{\tightlist}{%
  \setlength{\itemsep}{0pt}\setlength{\parskip}{0pt}}
\setcounter{secnumdepth}{5}
% Redefines (sub)paragraphs to behave more like sections
\ifx\paragraph\undefined\else
\let\oldparagraph\paragraph
\renewcommand{\paragraph}[1]{\oldparagraph{#1}\mbox{}}
\fi
\ifx\subparagraph\undefined\else
\let\oldsubparagraph\subparagraph
\renewcommand{\subparagraph}[1]{\oldsubparagraph{#1}\mbox{}}
\fi

%%% Use protect on footnotes to avoid problems with footnotes in titles
\let\rmarkdownfootnote\footnote%
\def\footnote{\protect\rmarkdownfootnote}

%%% Change title format to be more compact
\usepackage{titling}

% Create subtitle command for use in maketitle
\providecommand{\subtitle}[1]{
  \posttitle{
    \begin{center}\large#1\end{center}
    }
}

\setlength{\droptitle}{-2em}

  \title{Package as Analysis}
    \pretitle{\vspace{\droptitle}\centering\huge}
  \posttitle{\par}
    \author{Joshua H. Cook}
    \preauthor{\centering\large\emph}
  \postauthor{\par}
      \predate{\centering\large\emph}
  \postdate{\par}
    \date{2019-03-10}

\usepackage{booktabs}

\begin{document}
\maketitle

{
\setcounter{tocdepth}{1}
\tableofcontents
}
\chapter*{Welcome}\label{welcome}
\addcontentsline{toc}{chapter}{Welcome}

This is a manual on how to use the standard R package framework for data
analysis. Though potentially more work, especially at the start, the
purpose of using the R package framework is to maintain a clear and
reproducible analysis.

\textbf{{[}This book is currently in progress, though any
\href{https://github.com/jhrcook/package-as-analysis/issues}{feedback}
is welcome.{]}}

\section*{Why}\label{why}
\addcontentsline{toc}{section}{Why}

Why bother with maintaining a package framework while also doing data
analysis? It is a great question, especially when one considers the
complexity and fluidity of an analysis and the rigidity of the R package
framework. The answer is that some rigidity is needed - but just the
basics. That is what the R package framework provides. There is a place
for everything, though sometimes getting it to work (ie. build and pass
checks) requires a few extra steps.

During my own analyses, I found things were getting much too
disorganized and decentralized. What would start out as exploratory
would morph into a subdirectory graph more complex than the oringinal
parent analysis. Perhaps for more organized people, this is unnecessary,
but for those of us who want order and aren't sure how to get it, this
framework offers a great place to start.

The final (more abstract) reason for using the R package framework is to
battle the current issue of reproducibility. Reproducibility is the
cornerstone of science - if a finding it true, anyone should be able to
replicate it. However, the scientific community has been dealing with an
astounding amount of irreproducibility, most famously documented by the
\href{http://science.sciencemag.org/content/349/6251/aac4716}{Open
Science Collaboration}. If the analysis is organized as an R package,
though, an analysis can be re-run entirely by anyone else familliar with
R. Thus, whether they are collaborators and competitors, anyone should
be able to follow the analysis a scientist publishes.

\section*{Advantages}\label{advantages}
\addcontentsline{toc}{section}{Advantages}

There are many advantages to using this framework. Here are just a few,
though I am sure you will find there are many others:

\begin{itemize}
\tightlist
\item
  Because this is a standard framework, other will be able to navigate
  the directories and files adeptly.
\item
  The implementation of tests on functions and subroutines will make
  bugs easier to find and increase overall confidence in the validity of
  the analysis
\item
  This is a seamless mixture of scripts and markdown files for the
  separation of functions and analysis
\item
  Complete documentation of functions makes returning to code later much
  easier!
\item
  The analysis can take advantage of normal R package tools such as
  \href{https://travis-ci.org}{Travis-CL} and
  \href{https://codecov.io}{Codecov} integration,
  \href{https://pkgdown.r-lib.org}{pkgdown}, and
  \href{https://devtools.r-lib.org}{devtools} (build checks,
  documentation, etc.).
\end{itemize}

\section*{Examples}\label{examples}
\addcontentsline{toc}{section}{Examples}

{[}coming soon{]} Allele-specific \emph{KRAS} copy number alteration

\section*{About this Book}\label{about-this-book}
\addcontentsline{toc}{section}{About this Book}

{[}TODO{]}

\section*{About the Author}\label{about-the-author}
\addcontentsline{toc}{section}{About the Author}

I am a classically-trained biologist-turned computational biologist. I
graduated with degrees in Molecular Biology and Biochemistry, and
Chemistry from the University of California, Irvine in 2017. My research
focused on investigating the patterns and mechanisms of dissemination by
which \emph{Toxoplasma gondii}, an obligate, intracellular parasite,
infects a human host
\href{https://www.ncbi.nlm.nih.gov/pubmed/29295815}{Cook \emph{et al.},
2018}. I started my graduate studies at Harvard Medical School in 2018,
and after rotating in a chemical biology lab and a \emph{Vibrio
cholerae} lab, I finally decided to study cancer using computational
biology. Since then, and continuing still today, I have been learning
computer programming and statistics, trying to catch up to my peers.
Consequently, I have fallen in love with R, especially because of the
\href{https://www.tidyverse.org}{Tyidyverse} and tidy data.

\section*{Resources}\label{resources}
\addcontentsline{toc}{section}{Resources}

The best resource for making R packages is
\href{http://r-pkgs.had.co.nz}{R Packages} by Hadley Wickham.

There are some useful R packages you will want to have installed:

\begin{itemize}
\tightlist
\item
  \texttt{devtools} - will do most of the development building and
  checking
\item
  \texttt{roxygen2} - makes all of the documentation
\item
  \texttt{usethis} - for preparing all of the pieces and tools you want
  to include
\item
  \texttt{testthat} - for running tests
\item
  \texttt{kintr} - for compiling all of the Rmarkdown files
\end{itemize}

These can be installed using the following code.

\begin{Shaded}
\begin{Highlighting}[]
\KeywordTok{install.packages}\NormalTok{(}\KeywordTok{c}\NormalTok{(}\StringTok{"devtools"}\NormalTok{, }\StringTok{"roxygen2"}\NormalTok{, }\StringTok{"usethis"}\NormalTok{, }\StringTok{"testthat"}\NormalTok{, }\StringTok{"knitr"}\NormalTok{))}
\end{Highlighting}
\end{Shaded}

\section{License}\label{license}

This work is under a
\href{http://creativecommons.org/licenses/by-nc-sa/4.0/}{Attribution-NonCommercial-ShareAlike
4.0 International License (CC BY-NC-SA 4.0)}

\chapter{Getting Started}\label{getting-started}

Setting up the package is mostly automated and is well documented in
\href{http://r-pkgs.had.co.nz/intro.html}{R Packages} by Hadley Wickham.
If you are running the analysis on your local machine, I would recommend
using \href{https://www.rstudio.com}{RStudio} (which you likely already
do), but this is possible to do on a remote computing cluster (which is
how I work). I begin by going through the steps of setting up the basic
package framework, which is the same for local or remote work. Following
that, there is a section for how I work remotely. This can be skipped if
you only work locally or if you already have a system you enjoy (though
I highly recommend the system I currently use). I finish off with a few
extras that I recommend using, but are not necessary.

\section{Set-Up}\label{set-up}

\section{Remote}\label{remote}

\section{Extras}\label{extras}

\chapter{Framework}\label{framework}

Include a subsetion on each main peice of package:

\begin{itemize}
\tightlist
\item
  what is it for
\item
  things to be wary of
\item
  specific use for data analysis
\end{itemize}

\chapter{\texorpdfstring{Example: Allele-specific \emph{KRAS}
CNA}{Example: Allele-specific KRAS CNA}}\label{example-allele-specific-kras-cna}

{[}in progress{]}

\bibliography{book.bib}


\end{document}
